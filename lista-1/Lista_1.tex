% Options for packages loaded elsewhere
\PassOptionsToPackage{unicode}{hyperref}
\PassOptionsToPackage{hyphens}{url}
%
\documentclass[
  a4paper]{article}
\usepackage{amsmath,amssymb}
\usepackage{lmodern}
\usepackage{iftex}
\ifPDFTeX
  \usepackage[T1]{fontenc}
  \usepackage[utf8]{inputenc}
  \usepackage{textcomp} % provide euro and other symbols
\else % if luatex or xetex
  \usepackage{unicode-math}
  \defaultfontfeatures{Scale=MatchLowercase}
  \defaultfontfeatures[\rmfamily]{Ligatures=TeX,Scale=1}
\fi
% Use upquote if available, for straight quotes in verbatim environments
\IfFileExists{upquote.sty}{\usepackage{upquote}}{}
\IfFileExists{microtype.sty}{% use microtype if available
  \usepackage[]{microtype}
  \UseMicrotypeSet[protrusion]{basicmath} % disable protrusion for tt fonts
}{}
\makeatletter
\@ifundefined{KOMAClassName}{% if non-KOMA class
  \IfFileExists{parskip.sty}{%
    \usepackage{parskip}
  }{% else
    \setlength{\parindent}{0pt}
    \setlength{\parskip}{6pt plus 2pt minus 1pt}}
}{% if KOMA class
  \KOMAoptions{parskip=half}}
\makeatother
\usepackage{xcolor}
\usepackage[margin=1in]{geometry}
\usepackage{graphicx}
\makeatletter
\def\maxwidth{\ifdim\Gin@nat@width>\linewidth\linewidth\else\Gin@nat@width\fi}
\def\maxheight{\ifdim\Gin@nat@height>\textheight\textheight\else\Gin@nat@height\fi}
\makeatother
% Scale images if necessary, so that they will not overflow the page
% margins by default, and it is still possible to overwrite the defaults
% using explicit options in \includegraphics[width, height, ...]{}
\setkeys{Gin}{width=\maxwidth,height=\maxheight,keepaspectratio}
% Set default figure placement to htbp
\makeatletter
\def\fps@figure{htbp}
\makeatother
\setlength{\emergencystretch}{3em} % prevent overfull lines
\providecommand{\tightlist}{%
  \setlength{\itemsep}{0pt}\setlength{\parskip}{0pt}}
\setcounter{secnumdepth}{-\maxdimen} % remove section numbering
\ifLuaTeX
  \usepackage{selnolig}  % disable illegal ligatures
\fi
\IfFileExists{bookmark.sty}{\usepackage{bookmark}}{\usepackage{hyperref}}
\IfFileExists{xurl.sty}{\usepackage{xurl}}{} % add URL line breaks if available
\urlstyle{same} % disable monospaced font for URLs
\hypersetup{
  hidelinks,
  pdfcreator={LaTeX via pandoc}}

\author{}
\date{\vspace{-2.5em}}

\begin{document}

\begin{center}
{\Large
  PROGRAMA DE PÓS-GRADUAÇÃO EM COMPUTAÇÃO APLICADA} \\
\vspace{0.5cm}
\begin{figure}[!t]
\centering
\includegraphics[width=9cm, keepaspectratio]{logo-UnB.eps}
\end{figure}
\vskip 1em
{\large
  06 de novembro de 2022}
\vskip 3em
{\LARGE
  \textbf{Lista 1: geração de números pseudo-aleatórios}} \\
\vskip 1em
{\Large
  Prof. Guilherme Rodrigues} \\
\vskip 1em
{\Large
  Métodos computacionais intensivos para mineração de dados} \\
\vskip 1em
\end{center}

\vskip 5em

\begin{enumerate}
\item \textbf{As questões deverão ser respondidas em um único relatório \emph{PDF} ou \emph{html}, produzido usando as funcionalidades do \emph{Rmarkdown} ou outra ferramenta equivalente}.
\item \textbf{O aluno poderá consultar materiais relevantes disponíveis na internet, tais como livros, \emph{blogs} e artigos}.
\item \textbf{O trabalho é individual. Suspeitas de plágio e compartilhamento de soluções serão tratadas com rigor.}
\item \textbf{Os códigos \emph{R} utilizados devem ser disponibilizados na integra, seja no corpo do texto ou como anexo.}
\item \textbf{O aluno deverá enviar o trabalho até a data especificada na plataforma Microsoft Teams.}
\item \textbf{O trabalho será avaliado considerando o nível de qualidade do relatório, o que inclui a precisão das respostas, a pertinência das soluções encontradas, a formatação adotada, dentre outros aspectos correlatos.}
\item \textbf{Escreva seu código com esmero, evitando operações redundantes, visando eficiência computacional, otimizando o uso de memória, comentando os resultados e usando as melhores práticas em programação.}
\end{enumerate}

\newpage

\hypertarget{questuxe3o-1}{%
\subsection{\texorpdfstring{\textbf{Questão
1}}{Questão 1}}\label{questuxe3o-1}}

\textbf{Simulando computacionalmente o gerador de Babel.}

Todo seu destino, a cura do câncer e até o que vai acontecer no fim do
mundo. Todas essas respostas já estão escritas na Biblioteca de Babel.
Essa biblioteca proposta por Jorge Luís Borges é composta por um número
infinito de galerias, contendo todos os livros possíveis.

'' {[}\ldots{]} Um (livro) constava das letras M C V malevolamente
repetidas da primeira linha até a última. Outro é um simples labirinto
de letras mas a penúltima página diz `ó tempo tuas pirâmides'.''

A maior parte dos livros não tem qualquer significado. Entretanto,
embora improváveis, certos textos resultam em grandes obras, como o
Bhagavad Gita. Considerando as afirmações acima e a lista de palavras
existentes na língua portuguesa (disponível no arquivo
``Dicionario.txt''), responda aos itens a seguir.

\textbf{a)} Estime via simulação computacional (\emph{Monte Carlo}) a
probabilidade de se gerar uma palavra \emph{válida} (isso é, do
dicionário) ao sortear ao acaso sequências de 5 letras (todas com a
mesma probabilidade). Em seguida, calcule analiticamente tal
probabilidade e faça um gráfico indicando se a estimativa obtida se
aproxima do valor teórico conforme a amostra aumenta. \textbf{Atenção}:
utilize somente as letras do alfabeto sem carateres especiais.

\textbf{b)} Estime a probabilidade da sequência gerada ser um palíndromo
(ou seja, pode ser lida, indiferentemente, da esquerda para direita ou
da direita para esquerda). Compare o resultado com a probabilidade
exata, calculada analiticamente.

\textbf{c)} Construa um gerador que alterne entre consoantes e vogais
(se uma letra for uma vogal, a próxima será uma consoante e vice-versa).
Qual a probabilidade de gerar uma palavra válida com este novo gerador?

\textbf{d)} Considere um processo gerador de sequências de 5 caracteres
no qual cada letra é sorteada com probabilidade proporcional à sua
respectiva frequência na língua portuguesa (veja essa
\href{https://pt.wikipedia.org/wiki/Frequ\%C3\%AAncia_de_letras?wprov=sfla1}{página}).
Suponha que esse processo gerou uma sequência com ao menos um ``a''.
Neste caso, estime a probabilidade dessa sequência ser uma palavra
válida. \textbf{Dica}: Use a função \texttt{sample} e edite o parâmetro
\texttt{prob}. \textbf{Para pensar}: Você consegue calcular essa
probabilidade analiticamente? (Não precisa responder.)

\hypertarget{questuxe3o-2}{%
\subsection{\texorpdfstring{\textbf{Questão
2}}{Questão 2}}\label{questuxe3o-2}}

\textbf{Gerando números pseudo-aleatórios.}

\textbf{a)} Escreva uma função que gere, a partir do método da
transformada integral, uma amostra aleatória de tamanho \(n\) da
distribuição Cauchy para \(n\) e \(\gamma\) arbitrários. A densidade da
\(\text{Cauchy}(\gamma)\) é dada por
\[f(x)=\frac{1}{\pi \gamma (1 + (x/\gamma)^2)}.\] \textbf{Dica}: Veja
essa \href{https://en.wikipedia.org/wiki/Cauchy_distribution}{página}.

\vspace{.5cm}

\textbf{b)} Uma variável aleatória discreta \(X\) tem função massa de
probabilidade \begin{eqnarray*}
p(2)&=&0.2\\
p(3)&=&0.1\\
p(5)&=&0.2\\
p(7)&=&0.2\\
p(9)&=&0.3
\end{eqnarray*}

Use o método de transformação inversa para gerar uma amostra aleatória
de tamanho 1000 a partir da distribuição de \(X\). Construa uma tabela
de frequência relativa e compare as probabilidades empíricas com as
teóricas. Repita usando a função \emph{sample} do R.

\vspace{.5cm}

\textbf{c)} Escreva uma função que gere amostras da distribuição Normal
padrão (\(\mu=0, \sigma=1\)) usando o método de aceitação e rejeição
adotando como função geradora de candidatos, \(g(x)\), a distribuição
Cauchy padrão (isso é, com \(\gamma=1\)).

\end{document}
